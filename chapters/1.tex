\chapter{实验选题}

\section{基本信息}

\begin{itemize}
    \item 实验题目
    
    \textbf{魔塔}游戏的汇编设计与实现
    
    \item 编程语言
    
    x86 汇编语言
    
    \begin{itemize}
        \item 386 指令集
        
        使用 32 位汇编语言,但不启用浮点数运算指令
        
    \end{itemize}
    
    \item 外部库
    
    Win32 API
    
\end{itemize}

\section{实验的目的与意义}

为了巩固汇编语言与接口技术课程所学的理论知识,理解计算机的基本系统结构,理解处理器的工作过程,探究数据和指令的内部表述,以及\textbf{兴趣使然},特选择本实验题目。

训练利用汇编语言进行综合编程的能力,涉及数据结构、算法、图形输出、键盘输入等各个方面。实验选题是一个比较丰富、复杂、可玩性高的休闲益智策略游戏,因此十分考验汇编代码模块组织能力。

\section{实验的主要内容}

\begin{itemize}
    \item 用汇编语言实现“经典魔塔游戏”。
    \item 实现游戏场景切换。
    \item 游戏有存档功能。
    \item 需要从外部文件系统中获取图片资源。
    \item 可以使用键盘控制人物的行动、使用道具。
\end{itemize}

\section{实验要点}

\begin{itemize}
    \item 图形绘制:较高分辨率下的2D游戏图形界面绘制。
    \item 文件读写:游戏存档,资源导入。
    \item 输入控制:仅用键盘完成游戏。
    \item 算术运算:数值游戏过程中的数据计算。
    \item 模块编程:游戏比较复杂,需要精巧的程序设计方法来减小工作量。
\end{itemize}

\section{实验计划}

\begin{itemize}
    \item 2016 年 10 月:提交选题表
    \item 2016 年 11 月:设计与编程实现
    \item 2016 年 12 月:测试验收
\end{itemize}

\section{其他}

可能对“经典魔塔游戏”进行功能更新,例如加入渲染气氛的音乐,但不保证做 \footnote{实际上已经做了},尽量使得整个游戏过程没有违和感。

